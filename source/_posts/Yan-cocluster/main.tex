%%%
 % File: /main.tex
 % Created Date: Sunday September 10th 2023
 % Author: Zihan
 % -----
 % Last Modified: Thursday, 21st September 2023 11:01:00 am
 % Modified By: the developer formerly known as Zihan at <wzh4464@gmail.com>
 % -----
 % HISTORY:
 % Date      		By   	Comments
 % ----------		------	---------------------------------------------------------
%%%

\documentclass[12pt]{article}
\usepackage[a4paper, margin=0.8in]{geometry} % Adjust margins
\usepackage{setspace} % Adjust line spacing
% set spacing to 1.0
\setstretch{1.0} 
\usepackage{amsmath}
\usepackage{amssymb}
% \usepackage{mathptmx} 
% theorems
\usepackage{amsthm}


\title{Probability of detecting co-clusters and setting parameters}
\author{}
\date{}
\begin{document}
\maketitle
\section{Notation}
\begin{itemize}
    \item $A \in \mathbb{R}^{M \times N}$ is a matrix with $K$ co-clusters (co-cluster set $C = \{C_k\}_{k=1}^K$);
    \item $A$ is partitioned into $m \times n$ blocks, each block has size $\phi_i \times \psi_j$, that is, $M=\sum_{i=1}^m \phi_i$ and $N=\sum_{j=1}^n \psi_j$;
    \item thus block set $B = \{B_{(i,j)}\}_{i=1}^m,_{j=1}^n$;
          % \item $(i,j) \in \mathbb{N}/Q_m \times \mathbb{N}/Q_n$ is the index of a block;
          % sub-co-cluster $k$ 的落到block $(i,j)$ 的size 是 $M_i^{(k)} \times N_i^{(k)}$;
    \item the size of co-cluster $C_k \in \mathbb{R}^{M^{(k)} \times N^{(k)}}$ that falls into block $B_{(i,j)}$ is $M_{(i,j)}^{(k)} \times N_{(i,j)}^{(k)}$;
    \item $T_m$ is the minimum number of rows, $T_n$ is the minimum number of columns.
\end{itemize}

\section{Probability}
% P(Mi<Tm, Ni<Tn for all i in partition j | Qm, Qn, Ms, Ns)
Consider co-cluster $C_k$,
% P(M_{(i,j)}^{(k)} = i)
\begin{align*}
    P(M_{(i,j)}^{(k)} = \alpha) & = \frac{\binom{M^{(k)}}{\alpha} \binom{M-M^{(k)}}{\phi_i-\alpha}}{\binom{M}{\phi_i}} \\
    % p[x_] := Binomial[Mk, a] * Binomial[M-Mk, P-a] / Binomial[M, P]
    P(N_{(i,j)}^{(k)} = \beta)  & = \frac{\binom{N^{(k)}}{\beta} \binom{N-N^{(k)}}{\psi_j-\beta}}{\binom{N}{\psi_j}}
    % q[y_] := Binomial[Nk, b] * Binomial[N-Nk, Q-b] / Binomial[N, Q]
\end{align*}
The tail probability of $M_{(i,j)}^{(k)}$ and $N_{(i,j)}^{(k)}$ are
\begin{align*}
    P(M_{(i,j)}^{(k)} < T_m) & = \sum_{\alpha=1}^{T_m-1} P(M_{(i,j)}^{(k)} = \alpha) \\
                             & \le \exp(-2 (s_i^{(k)})^2 \phi_i)
\end{align*}
where $s_i^{(k)} = \cfrac{M^{(k)}}{M}-\cfrac{T_m-1}{\phi_i}$, and
\begin{align*}
    % &\le \sum_{\alpha=1}^{T_m-1} \frac{\binom{M^{(k)}}{\alpha} \binom{M-M^{(k)}}{\phi_i-\alpha}}{\binom{M}{\phi_i}} \\
    P(N_{(i,j)}^{(k)} < T_n) & = \sum_{\beta=1}^{T_n-1} P(N_{(i,j)}^{(k)} = \beta) \\
                             & \le \exp (-2 (t_j^{(k)})^2 \psi_j)
\end{align*}
where $t_j^{(k)} = \cfrac{N^{(k)}}{N}-\cfrac{T_n-1}{\psi_j}$.

The joint probability of $M_{(i,j)}^{(k)}$ and $N_{(i,j)}^{(k)}$ are
\begin{align*}
    P(M_{(i,j)}^{(k)} < T_m, N_{(i,j)}^{(k)} < T_n) & = \sum_{\alpha=1}^{T_m-1} \sum_{\beta=1}^{T_n-1} P(M_{(i,j)}^{(k)} = \alpha) P(N_{(i,j)}^{(k)} = \beta) \\
    % pq[x_, y_] := Sum[p[a] * q[b], {a, 1, x-1}, {b, 1, y-1}]
                                                    & \le \exp[-2 (s_i^{(k)})^2 \phi_i + -2 (t_j^{(k)})^2 \psi_j]
\end{align*}
If $\phi_i = p$ and $\psi_j = q$ for all $i$ and $j$, then

Suppose event $\omega_k$ is that co-cluster $C_k$ can't be find in any block $B_{(i,j)}$, then
\begin{align*}
    P(\omega_k) & = \prod_{i=1}^m \prod_{j=1}^n P(M_{(i,j)}^{(k)} < T_m, N_{(i,j)}^{(k)} < T_n)                          \\
                & \le \prod_{i=1}^m \prod_{j=1}^n \exp\{-2 \left[ (s_i^{(k)})^2 \phi_i + (t_j^{(k)})^2 \psi_j \right] \} \\
                & = \exp\{-2 \sum_{i=1}^m \sum_{j=1}^n \left[ (s_i^{(k)})^2 \phi_i + (t_j^{(k)})^2 \psi_j \right] \}     \\
\end{align*}

If $\phi_i = \phi$ and $\psi_j = \psi$ for all $i$ and $j$, then
\begin{align*}
    s_i^{(k)} & = s^{(k)} = \frac{M^{(k)}}{M}-\frac{T_m-1}{\phi} \\
    t_j^{(k)} & = t^{(k)} = \frac{N^{(k)}}{N}-\frac{T_n-1}{\psi}
\end{align*}

\begin{align*}
    P(\omega_k) & \le \exp \left\{ -2 [\phi m (s^{(k)})^2 + \psi n (t^{(k)})^2] \right\} \\
\end{align*}


And if we do $T_p$ times of random sampling, the Probability of detecting the co-cluster is
\begin{align*}
    P & = 1 - P(\omega_k)^{T_p}                                                        \\
      & \ge 1 - \exp \left\{ -2 T_p [\phi m (s^{(k)})^2 + \psi n (t^{(k)})^2] \right\} \\
\end{align*}
according to which, we can set $m, n, \phi, \psi, T_m, T_n$ and $T_p$ to ensure the probability of detecting the co-cluster is larger than a given threshold.

\section{Nosiy case}
\subsection{Assumption}
Assume each noise $n_{ij}$ complies with a normal distribution $N(0, \sigma^2)$, i.i.d. for all $i$ and $j$. Suppose $\exists \lambda > 0$, such that
$$\lambda \le \max(||B||_1, ||B^\top||_1)/\sigma^2.$$

\subsection{Score}
The score of a submatrix $A_{I,J}$ is defined as
\begin{align}
    S(I,J)       & = \min(S_{row}(I,J), S_{col}(I,J))                                                                                          \\
    S_{row}(I,J) & = \min_{i_1, i_2 \in I} \left(1- \frac{1}{|I|-1} \sum_{i_2 \in I, i_2 \neq i_1} \langle x_{i_1,J}, x_{i_2,J}\rangle \right) \\
    S_{col}(I,J) & = \min_{j_1, j_2 \in J} \left(1- \frac{1}{|J|-1} \sum_{j_2 \in J, j_2 \neq j_1} \langle x_{I,j_1}, x_{I,j_2}\rangle \right)
\end{align}
where $x_{i,J}$ is the $i$-th row of $A_{I,J}$. Here we define the inner product of two vectors $x$ and $y$ as
$$\langle x, y \rangle = \exp(-\frac{||x - y||_1^2}{2\alpha||x||_1||y||_1})$$

Suppose $A$ is a hidden co-cluster matrix, and $E$ is the noise matrix. Then the observed matrix is $B = A + E$. Consider co-cluster $B_{I,J}$, denote $1 - \frac{1}{|I|-1} \sum_{i_2 \in I, i_2 \neq i_1} \langle x_{i_1,J}, x_{i_2,J}\rangle$ as $s_{row}(i_1, i_2, J)$, then
\begin{align*}
    \mathbb{E}(s_{row}(i_1, i_2, J)) & = 1 - \frac{1}{|I|-1} \sum_{i_2 \in I, i_2 \neq i_1} \mathbb{E}(\langle x_{i_1,J}, x_{i_2,J}\rangle)                                    \\
                                     & = 1 - \frac{1}{|I|-1} \sum_{i_2 \in I, i_2 \neq i_1} \exp(-\frac{||x_{i_1,J} - x_{i_2,J}||_1^2}{2\alpha||x_{i_1,J}||_1||x_{i_2,J}||_1}) \\
                                     & \ge 1 - \exp(-\frac{2}{\alpha \min(||x_{i_1,J}||_1, ||x_{i_2,J}||_1)})                                                                  \\
                                     & \ge 1 - \exp(-\frac{2}{\alpha \max(||B||_1, ||B^\top||_1)})                                                                             \\
    \sigma^2(s_{row}(i_1, i_2, J))   & = \frac{1}{|I|-1} \sum_{i_2 \in I, i_2 \neq i_1} \sigma^2(\langle x_{i_1,J}, x_{i_2,J}\rangle)                                          \\
                                     & = \frac{1}{|I|-1} \sum_{i_2 \in I, i_2 \neq i_1} \exp(-\frac{||x_{i_1,J} - x_{i_2,J}||_1^2}{2\alpha||x_{i_1,J}||_1||x_{i_2,J}||_1})     \\
                                     & \le \sigma^2 \exp(-\frac{2}{\alpha \min(||x_{i_1,J}||_1, ||x_{i_2,J}||_1)})                                                             \\
                                     & \le \sigma^2 \exp(-\frac{2}{\alpha \max(||B||_1, ||B^\top||_1)})                                                                        \\
\end{align*}

Thus the expected value of $S_{row}(I,J)$ satisfies
\begin{align*}
    \mathbb{E}(S_{row}(I,J)) & \ge |J||I| \left(1 - \exp(-\frac{2}{\alpha \max(||B||_1, ||B^\top||_1)}) \right) \\
    \sigma^2(S_{row}(I,J))   & \le |I||J| \sigma^2 \exp(-\frac{2}{\alpha \max(||B||_1, ||B^\top||_1)})          \\
\end{align*}

Similarly, we can get
\begin{align*}
    \mathbb{E}(S_{col}(I,J)) & \ge |J||I| \left(1 - \exp(-\frac{2}{\alpha \max(||B||_1, ||B^\top||_1)}) \right) \\
    \sigma^2(S_{col}(I,J))   & \le |I||J| \sigma^2 \exp(-\frac{2}{\alpha \max(||B||_1, ||B^\top||_1)})          \\
\end{align*}

Then since $x^2 \le x$ for $x \in (0,1)$, we have
\begin{align*}
    \mathbb{E}(S(I,J)) & \ge 1 - \exp(-\frac{2}{\alpha \max(||B||_1, ||B^\top||_1)})             \\
    \sigma^2(S(I,J))   & \le |I||J| \sigma^2 \exp(-\frac{2}{\alpha \max(||B||_1, ||B^\top||_1)}) \\
\end{align*}

According to the Chernoff bound, we have
\begin{align*}
    P(S(I,J) \le \mathbb{E}(S(I,J)) - \epsilon)
     & \ge \exp(-\frac{\epsilon^2}{2|I||J| \sigma^2 \exp(-\frac{2}{\alpha \lambda})}) \\
\end{align*}

Thus if we set $\epsilon = \sqrt{2|I||J| \sigma^2 \exp(-\frac{2}{\alpha \lambda}) \log(1/\delta)}$, then

\begin{align*}
    P(S(I,J) \ge \mathbb{E}(S(I,J)) - \epsilon) & \ge \delta \\
\end{align*}

Combine with the probability control of $T_p$, we can select parameters to ensure the probability of detecting the co-cluster is larger than a given threshold.

\end{document}