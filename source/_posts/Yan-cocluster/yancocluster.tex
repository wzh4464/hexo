%%%
 % File: /yancocluster.tex
 % Created Date: Tuesday September 5th 2023
 % Author: Zihan
 % -----
 % Last Modified: Tuesday, 5th September 2023 9:06:53 am
 % Modified By: the developer formerly known as Zihan at <wzh4464@gmail.com>
 % -----
 % HISTORY:
 % Date      		By   	Comments
 % ----------		------	---------------------------------------------------------
%%%

\documentclass[12pt]{ctexart}
\usepackage[a4paper, margin=0.8in]{geometry} % Adjust margins
\usepackage{setspace} % Adjust line spacing
% set spacing to 1.0
\setstretch{1.0} 
\usepackage{amsmath}
\usepackage{amssymb}
% \usepackage{mathptmx} 
% theorems
\usepackage{amsthm}
\title{Co-clustering on large data arrays}
\author{}
\date{}
\begin{document}
\maketitle
\begin{itemize}
    \item 目标: 大规模数据数组的Co-clustering
    \item 策略: 将数据数组($\mathrm{2D}$ 的矩阵)分成小块,然后合并sub-co-clusters
    \item 问题: 划分后, co-cluster 可能会消失(不包含在任何 sub-co-clusters 中)
    \item 可能的解决方案: 以不同的方式进行划分并重复该过程;从统计学角度分析性能

    \item 假设:
          \begin{itemize}
              \item $A$ 是大小为 $M\times N$ 的输入矩阵
              \item $A$ 划分为 $Q_m\times Q_n$ 块,每个块的大小为 $B_m\times B_n$,即 $M=Q_m B_m$, $N = Q_n B_n$
              \item sub-co-cluster的阈值: 最小行数 $T_m$,最小列数 $T_n$
              \item 在块 $i$ 中,sub-co-cluster大小为 $M_i\times N_i$
              \item 我们执行 $T_p$ 次划分和重复检测过程
          \end{itemize}
    \item 分析:

          \begin{itemize}
              \item 现在考虑一个单个无噪声的co-cluster,大小为 $M_s\times N_s$: 如果我们至少有一个sub-co-cluster,其大小至少为 $T_m\times T_n$,那么我们可以通过递归检查行和列中的点来扩大它. 这一步非常简单. 然后我们可以获得大的co-cluster. 
              \item 问题是,如果我们在任何块中都没有检测到任何sub-co-cluster会发生什么. 也就是说,所有sub-co-cluster的大小都小于 $T_m\times T_n$. 

              \item 现在是时候开发有用的(和相关的)数学和统计公式了. 我们需要找到在任何块中都没有检测到任何sub-co-cluster的概率: $$P(M_i < T_m, N_i < T_n ; \forall i \in \pi_j | Q_m, Q_n, M_s, N_s)$$
              \item 我相信这可以通过组合获得. 我们可以以不同的方式 (例如,一个划分是 \\ $\{\{1,2,3\},\{4,5,6\}\}$,另一个是$\{\{1,3,5\},\{2,4,6\}\}$) 对矩阵进行划分,这可以通过随机采样完成. 现在假设划分是独立的,在所有划分中都没有检测到sub-co-cluster的概率是: $$\prod_{j=1}^{T_p} P(M_i < T_m, N_i < T_n ; \forall i \in \pi_j | Q_m, Q_n, M_s, N_s)$$
              \item 检测到co-cluster的概率是: $$1 - \prod_{j=1}^{T_p} P(M_i < T_m, N_i < T_n ; \forall i \in \pi_j | Q_m, Q_n, M_s, N_s)$$ 有了这个作为指导,我们会有想法来设置 $T_m$, $T_n$, $Q_m$, $Q_n$ 和 $T_p$. 
          \end{itemize}
    \item 任务:
          \begin{itemize}

              \item 寻找概率
              \item 检查理论对单个无噪声co-cluster是否正确, co-cluster有很多不同的大小
              \item 嵌入更多无噪声co-cluster
              \item 尝试嘈杂co-cluster
              \item 在模拟和真实例子中试试
          \end{itemize}
\end{itemize}

我认为如果您能检测到大量椭圆或者找到另一个应用,就可以写一篇好论文. 

如我所说,要发表一篇好论文,您需要一个好的想法、好的描述和令人印象深刻的结果. 我相信我们已经有了第一个,但我们还需要后面的$\mathrm{2}$和$\mathrm{3}$. 

\end{document}